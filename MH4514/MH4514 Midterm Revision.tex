\documentclass[12pt]{extarticle}
\usepackage{tcolorbox}
\tcbuselibrary{skins} %preamble
\usepackage{tabularx}
%Some packages I commonly use.
\usepackage[english]{babel}
\usepackage{graphicx}
\usepackage{framed}
\usepackage[normalem]{ulem}
\usepackage{amsmath}
\usepackage{amsthm}
\usepackage{amssymb}
\usepackage{amsfonts}
\usepackage{enumerate}
\usepackage[utf8]{inputenc}
\usepackage[top=1 in,bottom=1in, left=1 in, right=1 in]{geometry}

%A bunch of definitions that make my life easier
\newcommand{\matlab}{{\sc Matlab} }
\newcommand{\cvec}[1]{{\mathbf #1}}
\newcommand{\rvec}[1]{\vec{\mathbf #1}}
\newcommand{\ihat}{\hat{\textbf{\i}}}
\newcommand{\jhat}{\hat{\textbf{\j}}}
\newcommand{\khat}{\hat{\textbf{k}}}
\newcommand{\minor}{{\rm minor}}
\newcommand{\trace}{{\rm trace}}
\newcommand{\spn}{{\rm Span}}
\newcommand{\rem}{{\rm rem}}
\newcommand{\ran}{{\rm range}}
\newcommand{\range}{{\rm range}}
\newcommand{\mdiv}{{\rm div}}
\newcommand{\proj}{{\rm proj}}
\newcommand{\R}{\mathbb{R}}
\newcommand{\N}{\mathbb{N}}
\newcommand{\Q}{\mathbb{Q}}
\newcommand{\Z}{\mathbb{Z}}
\newcommand{\<}{\langle}
\renewcommand{\>}{\rangle}
\renewcommand{\emptyset}{\varnothing}
\newcommand{\attn}[1]{\textbf{#1}}
\theoremstyle{definition}
\newtheorem{theorem}{Theorem}
\newtheorem{corollary}{Corollary}
\newtheorem*{definition}{Definition}
\newtheorem*{example}{Example}
\newtheorem*{note}{Note}
\newtheorem{exercise}{Exercise}
\newcommand{\bproof}{\bigskip {\bf Proof. }}
\newcommand{\eproof}{\hfill\qedsymbol}
\newcommand{\Disp}{\displaystyle}
\newcommand{\qe}{\hfill\(\bigtriangledown\)}
\setlength{\columnseprule}{1 pt}


\title{\textbf{MH4514 Financial Mathematics}\\
\Large - Midterm Revision -}
\author{Naoki Honda}
\date{March 2019}

\begin{document}

\maketitle


\section{(10.6) Properties of the Conditional Expectation}
\begin{tcolorbox}[enhanced, drop fuzzy shadow, title=Properties of CE]
\begin{itemize}
    \item   If $G$ depends only on the information contained in $\mathcal{G}$,
            \begin{align*}
            \mathbb{E}[FG|\mathcal{G}]=G\mathbb{E}[F|\mathcal{G}]
            \end{align*}
    \item   If $G$ depends only on the information contained in $\mathcal{G}$,
            \begin{align*}
            \mathbb{E}[G|\mathcal{G}]=G
            \end{align*}
    \item   \textbf{Tower Property}
            \begin{align*}
                \mathbb{E}[\mathbb{E}[F|\mathcal{G}]|\mathcal{H}] = \mathbb{E}[F|\mathcal{H}] & \text{\ \ \ \ (if\ } \mathcal{H} \subset \mathcal{G})
            \end{align*}
    \item   When $F$ "\underline{does not depend}" on the information
            contained in $\mathcal{G}$ (more stated, when the random variable $F$ is \textit{independent} of the $\sigma$-algebra $\mathcal{G}$),
            \begin{align*}
             \mathbb{E}[F|\mathcal{G}]=\mathbb{E}[F]
            \end{align*}
    \item   If $G$ depends only on $\mathcal{G}$ 
            and $F$ is independent of $\mathcal{G}$, then
            \begin{align*}
            \mathbb{E}[h(F,G)|\mathcal{G}]=\mathbb{E}[h(F,x)]_{x=G}
            \end{align*}
\end{itemize}
\end{tcolorbox}

\newpage
\section{(1) Discrete-Time Martingales}
\subsection{Definition}
\begin{tcolorbox}[enhanced, drop fuzzy shadow, title=Definition 1.1]
An integrable\footnote{Integrable means $\mathbb{E}[|Z_n|] < \infty$ for all $n \in \mathbb{N}$}, discrete-time process $(Z_n)_{n \in \mathbb{N}}$ is a \underline{martingale} with respect to $(\mathcal{F}_n)_{n \in \mathbb{N}}$ if $(Z_n)_{n \in \mathbb{N}}$ is $\mathcal{F}_n$-adapted and satisfies the property
\begin{align*}
    && \mathbb{E}[Z_{n+1} | \mathcal{F}_n] = Z_n, && n \in \mathbb{N}
\end{align*}
\end{tcolorbox}

\subsection{Properties}
A particular property of martingales is that their expectation is constant over time.
\begin{tcolorbox}[enhanced, drop fuzzy shadow, title=Proposition 1.2]
Let $(Z_n)_{n \in \mathbb{N}}$ be martingale, we have
\begin{align*}
    && \mathbb{E}[Z_{n}] = \mathbb{E}[Z_{0}], && n \in \mathbb{N}
\end{align*}
\end{tcolorbox}
\ \\ 
The following Theorem 1.6 is called the \underline{Doob's stopping time theorem}.
\begin{tcolorbox}[enhanced, drop fuzzy shadow, title=Theorem 1.6]
Assume that $(M_n)_{n \in \mathbb{N}}$ is a martingale with respect to $(\mathcal{F}_n)_{n \in \mathbb{N}}$. Then the \textit{stopped process} $(M_{\tau \wedge n})_{n \in \mathbb{N}}$ is also a martingale with respect to $(\mathcal{F}_n)_{n \in \mathbb{N}}$.
\end{tcolorbox}



\subsection{Applications: Ruin Probability}
Consider the standard random walk (or gambling process) $(S_n)_{n \in \mathbb{N}}$ on $\{0,1,...,B\}$ with independent $\{-1,1\}$-valued increments, and
\begin{align*}
    \mathbb{P}(S_{n+1}-S_n = +1)=p && \text{and} && \mathbb{P}(S_{n+1}-S_n = -1)=q, && n \in \mathbb{N}
\end{align*}
Let
\begin{align*}
    T_{0,B}:\Omega \longrightarrow \mathbb{N}
\end{align*}
be the first hitting time of the boundary $\{0,B\}$, define stopping time $\tau$ by
\begin{align*}
    \tau := T_{0,B} := \text{inf}\{n \geq 0: S_n = B \text{\ or\ } S_n = 0\}
\end{align*}

\newpage
We will recover the ruin probabilities
\begin{align*}
    \mathbb{P}(s_\tau =0 | S_0 = k), && k=0,1,...,B
\end{align*}
First in the unbiased case $p=q=1/2$

\subsubsection{Unbiased Ruin Probability with Martingale}
\underline{Step 1}. The process $(S_n)_{n \in \mathbb{N}}$ is a martingale. \\ \\
We note that the process $(S_n)_{n \in \mathbb{N}}$ has independent increments, and in the unbiased case $p=q=1/2$ those increments are centered:
\begin{align*}
    \mathbb{E}[S_{n+1} - S_n]=1 \times p + (-1) \times q = 1 \times \frac{1}{2} + (-1) \times \frac{1}{2} =0,
\end{align*}
hence $(S_n)_{n \in \mathbb{N}}$ is a \textit{martingale}.\\ \\
\underline{Step 2}. The stopped process $(S_{\tau \wedge n})_{n \in \mathbb{N}}$ is also a martingale, as a consequence of Theorem 1.6 (Doob's stopping time theorem)\\ \\
\underline{Step 3}. Since the stopped process $(S_{\tau \wedge n})_{n \in \mathbb{N}}$ is a martingale, we find that its expectation $\mathbb{E}[S_{\tau \wedge n}|S_0 = k]$ is constant in $n \in \mathbb{N}$ by proposition 1.2, which gives
\begin{align*}
        k=\mathbb{E}[S_0 |S_0=k]=\mathbb{E}[S_{\tau \wedge n} |S_0=k], && k=0,1,...,B
\end{align*}
Letting $n$ go to infinity we get
\begin{align*}
    \mathbb{E}[S_\tau | S_0 =k] &= \mathbb{E}\left[\lim_{n \to \infty} S_{\tau \wedge n} | S_0 =k\right] \\
    &= \lim_{n \to \infty} \mathbb{E}[S_{\tau \wedge n} | S_0 =k] =k
\end{align*}
where the exchange between limit and expectation is justified by the boundedness $|S_{\tau \wedge n}| \leq B$ \textit{a.s.}, $n \in \mathbb{N}$.\\
Hence we have
\begin{align*}
    \begin{cases}
        0 \times \mathbb{P}(S_\tau =0|S_0 =k) + B \times \mathbb{P}(S_\tau =B|S_0 =k) =\mathbb{E}[S_\tau | S_0 =k] =k \\
        \mathbb{P}(S_\tau =0|S_0 =k) + \mathbb{P}(S_\tau =B|S_0 =k) = 1
    \end{cases}
\end{align*}
which shows that
\begin{align*}
    \mathbb{P}(S_\tau =B|S_0 =k)=\frac{k}{B} && \text{and} && \mathbb{P}(S_\tau =0|S_0 =k)=1-\frac{k}{B}
\end{align*}

\newpage
\subsubsection{Biased Ruin Probability with Martingale}
Next, for the unbiased case where $p \neq q$.
Here we note that the processs
\begin{align*}
    && M_n := \left( \frac{q}{p} \right)^{S_n}, && n \in \mathbb{N}
\end{align*}
is a martingale with respect to $(\mathcal{F}_n)_{n \in \mathbb{N}}$.\\
\underline{Step 1}. The process $(M_n)_{n \in \mathbb{N}}$ is a martingale.
\begin{align*}
    \mathbb{E}[M_{n+1}|\mathcal{F}_n]&= \mathbb{E}\left[\left( \frac{q}{p} \right)^{S_{n+1}} \bigg| \mathcal{F}_n \right] = \mathbb{E}\left[\left( \frac{q}{p} \right)^{S_{n+1} -S_n} \left( \frac{q}{p} \right)^{S_n} \bigg| \mathcal{F}_n \right] \\
    &= \left( \frac{q}{p} \right)^{S_n} \mathbb{E}\left[\left( \frac{q}{p} \right)^{S_{n+1} -S_n} \bigg| \mathcal{F}_n \right]\\
    &= \left( \frac{q}{p} \right)^{S_n} \mathbb{E}\left[\left( \frac{q}{p} \right)^{S_{n+1} -S_n} \right]\\
    &= \left( \frac{q}{p} \right)^{S_n} \left( \frac{q}{p} \mathbb{P}(S_{n+1} -S_{n} =1) + \left(\frac{q}{p}\right)^{-1} \mathbb{P}(S_{n+1} -S_{n} =-1) \right)\\
    &= \left( \frac{q}{p} \right)^{S_n} \left( p\frac{q}{p}  + q\left(\frac{q}{p}\right)^{-1} \right)\\
    &= \left( \frac{q}{p} \right)^{S_n} (q+p) = \left( \frac{q}{p} \right)^{S_n} = M_n
\end{align*}
$n \in \mathbb{N}$. In particular, the expectation of $(M_n)_{n \in \mathbb{N}}$ is constant over time by Proposition 1.2 since it is a martingale, \textit{i.e.} we have
\begin{align*}
    && \left( \frac{q}{p} \right)^k = \mathbb{E}[M_0 | S_0 =k] = \mathbb{E}[M_n | S_0 =k], && k=0,1,...,B, && n \in \mathbb{N}
\end{align*}
\underline{Step 2}. The stopped process $(M_{\tau \wedge n})_{n \in \mathbb{N}}$ is also a martingale, as a consequence of Theorem 1.6\\ \\
\underline{Step 3}. Since the stopped process $(M_{\tau \wedge n})_{n \in \mathbb{N}}$ remains a martingale, its expected value $\mathbb{E}[M_{\tau \wedge n}|S_0 =k]$ is constant in $n \in \mathbb{N}$ by Proposition 1.2, this gives
\begin{align*}
    \left( \frac{q}{p} \right)^k = \mathbb{E}[M_0 | S_0 =k] = \mathbb{E}[M_{\tau \wedge n} | S_0 =k]
\end{align*}
Next, letting $n$ go to infinity we find
\begin{align*}
    \left( \frac{q}{p} \right)^k &= \mathbb{E}[M_0 | S_0 =k] = \lim_{n \to \infty}\mathbb{E}[M_{\tau \wedge n} | S_0 =k]\\
    &= \mathbb{E}[\lim_{n \to \infty}M_{\tau \wedge n} | S_0 =k]\\
    &= \mathbb{E}[M_\tau | S_0 =k]
\end{align*}
hence
\begin{align*}
    \left( \frac{q}{p} \right)^k &= \mathbb{E}[M_\tau | S_0 =k]\\
    &= \left( \frac{q}{p} \right)^B \mathbb{P}\left(M_\tau = \left( \frac{q}{p} \right)^B \bigg| S_0 =k \right) + \left( \frac{q}{p} \right)^0 \mathbb{P}\left(M_\tau = \left( \frac{q}{p} \right)^0 \bigg| S_0 =k \right)\\
    &= \left( \frac{q}{p} \right)^B \mathbb{P}\left(M_\tau = \left( \frac{q}{p} \right)^B \bigg| S_0 =k \right) + \mathbb{P}(M_\tau = 1 | S_0 =k )
\end{align*}
Solving the system of equations
\begin{align*}
    \begin{cases}
    \left( \frac{q}{p} \right)^k = \left( \frac{q}{p} \right)^B \mathbb{P}\left(M_\tau = \left( \frac{q}{p} \right)^B \bigg| S_0 =k \right) + \mathbb{P}(M_\tau = 1 | S_0 =k )\\
    \mathbb{P}\left(M_\tau = \left( \frac{q}{p} \right)^B \bigg| S_0 =k \right) + \mathbb{P}(M_\tau = 1 | S_0 =k ) =1
    \end{cases}
\end{align*}
gives
\begin{align*}
    \mathbb{P}(S_\tau = B | S_0 =k ) &= \mathbb{P}\left(M_\tau = \left( \frac{q}{p} \right)^B \bigg| S_0 =k \right)\\
    &= \frac{(q/p)^k -1}{(q/p)^B -1}
\end{align*}
and
\begin{align*}
    \mathbb{P}(S_\tau = 0 | S_0 =k ) &= \mathbb{P}(M_\tau = 1 | S_0 =k)\\
    &= 1- \frac{(q/p)^k -1}{(q/p)^B -1}\\
    &= \frac{(q/p)^B - (q/p)^k}{(q/p)^B -1}
\end{align*}
for $k=0,1,...,B$.

\newpage
\subsection{Applications: Mean Game Duration}
In this section we show how we can recover the mean game duration $\mathbb{E}[\tau | S_0 =k]$.\\ \\
\subsubsection{Unbiased Mean Game Duration with Martingale}
In the case of a fair game $p=q=1/2$,\\
\underline{Step 1}. The process $(S_n^2 -n)_{n \in \mathbb{N}}$ is a martingale. \\
\begin{align*}
    \mathbb{E}[S_{n+1}^2 -(n+1)|\mathcal{F}_n] &= \mathbb{E}[(S_n + S_{n+1} - S_n)^2 -(n+1)|\mathcal{F}_n]\\
    &= \mathbb{E}[S_n^2 + (S_{n+1} - S_n)^2 +2S_n(S_{n+1} - S_n)-(n+1)|\mathcal{F}_n]\\
    &= \mathbb{E}[S_n^2-n-1|\mathcal{F}_n] + \mathbb{E}[(S_{n+1} - S_n)^2|\mathcal{F}_n] +2\mathbb{E}[S_n(S_{n+1} - S_n)|\mathcal{F}_n]\\
    &= S_n^2-n-1 +\mathbb{E}[(S_{n+1} - S_n)^2] +2S_n\mathbb{E}[S_{n+1} - S_n]\\
    &= S_n^2-n-1 +\mathbb{E}[(S_{n+1} - S_n)^2]\\
    &= S_n^2-n
\end{align*}
\underline{Step 2}. The stopped process $(S_{\tau \wedge n}^2 -\tau \wedge n)_{n \in \mathbb{N}}$ is also a martingale, as a consequence of Theorem 1.6\\ \\
\underline{Step 3}. Since the stopped process $(S_{\tau \wedge n}^2 -\tau \wedge n)_{n \in \mathbb{N}}$ is also a martingale, its expectation $\mathbb{E}[S_{\tau \wedge n}^2 -\tau \wedge n | S_0 =k]$ is constant in $n \in \mathbb{N}$ by Proposition 1.2, hence we have
\begin{align*}
    k^2 = \mathbb{E}[S_0^2 -0 | S_0 =k] = \mathbb{E}[S_{\tau \wedge n}^2 -\tau \wedge n | S_0 =k]
\end{align*}
and after taking the limit as $n$ tends to infinity,
\begin{align*}
    k^2 &=\lim_{n \to \infty}\mathbb{E}[S_{\tau \wedge n}^2 -\tau \wedge n | S_0 =k]\\
    &=\lim_{n \to \infty}\mathbb{E}[S_{\tau \wedge n}^2| S_0 =k] -\lim_{n \to \infty}\mathbb{E}[\tau \wedge n| S_0 =k]\\
    &= \mathbb{E}\left[\lim_{n \to \infty}S_{\tau \wedge n}^2| S_0 =k\right] -\mathbb{E}\left[\lim_{n \to \infty} \tau \wedge n| S_0 =k\right]\\
    &= \mathbb{E}\left[\lim_{n \to \infty}S_{\tau \wedge n}^2 - \lim_{n \to \infty} \tau \wedge n| S_0 =k\right]\\
    &= \mathbb{E}[S_\tau^2 - \tau| S_0 =k]
\end{align*}
since $S_{\tau \wedge n}^2 \in [0, B^2]$ for all $n \in \mathbb{N}$ and $n \mapsto \tau \wedge n$ is nondecreasing, and this gives
\begin{align*}
    k^2 &= \mathbb{E}[S_\tau^2 - \tau| S_0 =k]\\
    &= \mathbb{E}[S_\tau^2| S_0 =k] - \mathbb{E}[\tau| S_0 =k]\\
    &= B^2 \mathbb{P}(S_\tau =B|S_0 = k) + 0^2 \mathbb{P}(S_\tau =0|S_0 = k) - \mathbb{E}[\tau| S_0 =k]
\end{align*}
\textit{i.e.}
\begin{align*}
    \mathbb{E}[\tau| S_0 =k] &= B^2(S_\tau =B|S_0 = k)-k^2\\
    &= B^2 \frac{k}{B} -k^2\\
    &= k(B-k), & k=0,1,...,B
\end{align*}

\newpage
\subsubsection{Biased Mean Game Duration with Martingale}
In the case of non-symmetric case where $p \neq q$\\
\underline{Step 1}. The process $S_n -(p-q)n$ is a martingale. \\
\begin{align*}
    \mathbb{E}[S_n -S_{n-1} -(p-q)] = \mathbb{E}[S_n - S_{n-1}] -(p-q) =0
\end{align*}
\underline{Step 2}. The stopped process $(S_{\tau \wedge n} -(p-q)(\tau \wedge n))_{n \in \mathbb{N}}$ is also a martingale, as a consequence of Theorem 1.6\\ \\
\underline{Step 3}. The expectation $\mathbb{E}[S_{\tau \wedge n} -(p-q)(\tau \wedge n) | S_0 =k]$ is constant in $n \in \mathbb{N}$\\ \\
\underline{Step 4}. Since the stopped process $(S_{\tau \wedge n} -(p-q)(\tau \wedge n))_{n \in \mathbb{N}}$ is a martingale, we have
\begin{align*}
    k = \mathbb{E}[S_0 -0 | S_0 =k] = \mathbb{E}[S_{\tau \wedge n} -(p-q)(\tau \wedge n)| S_0 =k]
\end{align*}
and after taking the limit as $n$ tends to infinity,
\begin{align*}
    k &= \lim_{n \to \infty} \mathbb{E}[S_{\tau \wedge n} -(p-q)(\tau \wedge n)| S_0 =k]\\
    &= \mathbb{E}[\lim_{n \to \infty}S_{\tau \wedge n} -(p-q)\lim_{n \to \infty}\tau \wedge n| S_0 =k]\\
    &= \mathbb{E}[S_\tau -(p-q)\tau| S_0 =k]
\end{align*}
which gives
\begin{align*}
    k &= \mathbb{E}[S_\tau -(p-q)\tau| S_0 =k]\\
    &= \mathbb{E}[S_\tau| S_0 =k]-(p-q)\mathbb{E}[\tau| S_0 =k]\\
    &= B \times \mathbb{P}(S_\tau =B| S_0 =k) + 0\times \mathbb{P}(S_\tau =0| S_0 =k) - (p-q)\mathbb{E}[\tau| S_0 =k]
\end{align*}
\textit{i.e.}
\begin{align*}
    (p-q)\mathbb{E}[\tau| S_0 =k] &= B \times \mathbb{P}(S_\tau =B| S_0 =k)-k\\
    &= B\frac{(q/p)^k-1}{(q/p)^B-1}-k,
\end{align*}
hence
\begin{align*}
    \mathbb{E}[\tau| S_0 =k]=\frac{1}{p-q}\left(B\frac{(q/p)^k-1}{(q/p)^B-1}-k \right)
\end{align*}
for $k=0,1,...,B$ \\

\newpage
\subsection{Applications: Summary}
Here's the summarized table of the family of martingales.\\ 
\begin{tcolorbox}[enhanced, drop fuzzy shadow, title=Table 1.1: List of martingales.]
\begin{center}
  \begin{tabularx}{6in}{| X | X | X |} \hline
    \  & Unbiased & Biased \\ \hline
    Ruin probability & $S_n$ & $\left(\frac{q}{p}\right)^{S_n}$ \\ \hline
    Mean game duration & $S_n^2 -n$ & $S_n -n(p-q)$ \\ \hline
  \end{tabularx}
\end{center}
\end{tcolorbox}
\newpage

\section{(2) Assets, Portfolios, and Arbitrage}
\subsection{(0) Back to the Introduction}
\begin{tcolorbox}[enhanced, drop fuzzy shadow, title=Definition 0.1: Put option]
A (European) \textit{put} option is a contract that gives its holder the right to \textit{sell} a quantity of assets at a predefined price $K$ called the strike price and at a predefined date $T$ called maturity
\end{tcolorbox}
In general, the payoff of a put option takes the form
\begin{align*}
    \phi(S_T)=(K-S_T)^+ :=
    \begin{cases}
    K-S_T  & S_T \leq K \\
    0  & S_T \geq K
    \end{cases}
\end{align*}
\begin{tcolorbox}[enhanced, drop fuzzy shadow, title=Definition 0.2: Call option]
A (European) \textit{call} option is a contract that gives its holder the right to \textit{buy} a quantity of assets at a predefined price $K$ called the strike price and at a predefined date $T$ called maturity
\end{tcolorbox}
In general, the payoff of a put option takes the form
\begin{align*}
    \phi(S_T)=(S_T-K)^+ :=
    \begin{cases}
    0  & S_T \leq K \\
    S_T-K  & S_T \geq K
    \end{cases}
\end{align*}

\begin{tcolorbox}[enhanced, drop fuzzy shadow, title=Definition 0.3: Pricing and hedging in a binary model]
The \textit{arbitrage price} of the option is interpreted as the initial cost $\alpha S_0 + \$\beta$ of the portfolio hedging the claim $\mathcal{C}$
\end{tcolorbox}

\newpage
\subsection{(2.4) Risk-Neutral Probability Measures}
Bunch of theorem below...
\begin{tcolorbox}[enhanced, drop fuzzy shadow, title=Theorem 2.5]
A market is \textit{without} arbitrage opportunity $\iff$ It admits at least one (equivalent\footnote{Referring to Definition 2.4; A probability measure $\mathbb{P}^*$ on $(\Omega, \mathcal{F})$ is said to be \textit{equivalent} to another probability measure $\mathbb{P}$ when $\mathbb{P}^*(A)=0 \iff \mathbb{P}(A)=0$ for all $A \in \mathcal{F}$}) risk-neutral probability measure $\mathbb{P}^*$
\end{tcolorbox}

\subsection{(2.6) Market Completeness}
\begin{tcolorbox}[enhanced, drop fuzzy shadow, title=Definition 2.10: Pricing and hedging in a binary model]
A market model is said to be \textit{complete} if every contingent claim $\mathcal{C}$ is attainable.
\end{tcolorbox}

\begin{tcolorbox}[enhanced, drop fuzzy shadow, title=Definition 2.11: Pricing and hedging in a binary model]
A market model without arbitrage opportunities is complete $\iff$ It admits only one (equivalent) risk-neutral probability measure $\mathbb{P}^*$.
\end{tcolorbox}

\newpage
\section{(3) Discrete-Time Model}
\subsection{(3.5) Multingales and Conditional Expectation}
\begin{tcolorbox}[enhanced, drop fuzzy shadow, title=Definition 3.4]
A stochastic process $(M_t)_{t=0,1,...,N}$ is called a discrete-time \underline{martingale} with respect to the filtration $(\mathcal{F}_t)_{t=0,1,...,N}$ if $(M_t)_{t=0,1,...,N}$ is $(\mathcal{F}_t)_{t=0,1,...,N}$-adapted and satisfies the property
\begin{align*}
    && \mathbb{E}[M_{t+1} | \mathcal{F}_t] = M_t, && t=0,1,...,N-1.
\end{align*}
\end{tcolorbox}
\ \\
\begin{tcolorbox}[enhanced, drop fuzzy shadow, title=Proposition 3.5]
Let $(Z_n)_{n \in \mathbb{N}}$ be martingale, we have
\begin{align*}
    && \mathbb{E}[Z_{n}] = \mathbb{E}[Z_{0}], && n \in \mathbb{N}
\end{align*}
\end{tcolorbox}

\subsection{(3.6) Market Completeness and Risk-Neutral Measures}
\begin{tcolorbox}[enhanced, drop fuzzy shadow, title=Definition 3.8]
A probability measure $\mathbb{P}^*$ on $\Omega$ is called a rick-neutral measure if
\begin{align*}
    && \mathbb{E}^*\left[S_{t+1}^{(i)}\bigg| \mathcal{F}_t \right]=(1+r)S_{t}^{(i)}, && i=1,2,...,d
\end{align*}
\end{tcolorbox}
\ \\
\begin{tcolorbox}[enhanced, drop fuzzy shadow, title=Definition 3.9]
A probability measure $\mathbb{P}^*$ on $\Omega$ is called a rick-neutral measure if
\begin{align*}
    && \Tilde{S}_{t}^{(i)}:= \frac{S_t^{(i)}}{(1+r)^t}, && t=0,1,...,n
\end{align*}
is a martingale under $\mathbb{P}^*$, \textit{i.e.}
\begin{align*}
    && \mathbb{E}^*\left[\Tilde{S}_{t+1}^{(i)}\bigg| \mathcal{F}_t \right]=\Tilde{S}_{t}^{(i)}, && t=0,1,...,N-1
\end{align*}
$i=0,1,...,d$.
\end{tcolorbox}


\newpage
\subsection{(3.7) The Cox-Ross-Rubinstein (CRR) Market Model}
Consider the portfolio consists of
\begin{itemize}
    \item Risk-free asset priced as
    \begin{align*}
    S_t^{(0)} =S_0^{(0)} (1+r)^t, && t=0,1,...,N
    \end{align*}
    \item Risky asset priced as
    \begin{align*}
        S_t^{(1)} = S_0^{(1)}\prod^t_{k=1}(1+R_k), && t=0,1,...,N
    \end{align*}
    where $R_t \in \{a,b \}$,     $t=1,2,...,N$
\end{itemize}
\begin{tcolorbox}[enhanced, drop fuzzy shadow, title=Theorem 3.15]
The CRR model is without arbitrage opportunities if and only if $a<r<b$. In this case the market is complete and the (equivalent) risk-neutral probability measure $\mathbb{P}^*$ is given by
\begin{align*}
    \mathbb{P}^*(R_{t+1}=b|\mathcal{F}_t)=\frac{r-a}{b-a} \text{\ \ \ \ and\ \ \ \ }\mathbb{P}^*(R_{t+1}=a|\mathcal{F}_t)=\frac{b-r}{b-a}
\end{align*}
$t=0,1,...,N-1$. In particular, $(R_1,R_2,...R_N)$ forms a sequence of \textit{i.i.d} random variables under $\mathbb{P}^*$, with
\begin{align*}
    p^* := \mathbb{P}^*(R_{t}=b)=\frac{r-a}{b-a} \text{\ \ \ \ and\ \ \ \ } q^* := \mathbb{P}^*(R_{t}=a)=\frac{b-r}{b-a}
\end{align*}
\end{tcolorbox}



\newpage
\section{(4) Pricing and Hedging in Discrete Time}
\subsection{(4.1) Pricing of Contingent Claims}
\begin{tcolorbox}[enhanced, drop fuzzy shadow, title=Lemma 4.2]
The following statements are equivalent:
\begin{itemize}
    \item (i) The portfolio strategy $(\Bar{\xi}_t)_{t=1,2,...,N}$ is self-financing \\
    \item (ii) $\Bar{\xi}_t \cdot \Bar{X}_t = \Bar{\xi}_{t+1} \cdot \Bar{X}_t$ for all $t=1,2,...,N-1$. \\
    \item (iii) The discounted portfolio value $\Tilde{V}_t$ can be written as the stochastic summation
    \begin{align*}
        \Tilde{V}_t = \Tilde{V}_0 + \underbrace{\sum^t_{k=1}\Bar{\xi}_k \cdot (\Bar{X}_k - \Bar{X}_{k-1})}_\text{sum of profits and losses}, && t=0,1,...,N,
    \end{align*}
    of discounted profits and losses.
\end{itemize}
\end{tcolorbox}
\ \\
\begin{tcolorbox}[enhanced, drop fuzzy shadow, title=Theorem 4.5]
The arbitrage price $\pi_t (C)$ of an attainable conntingent claim $C$ is given by
\begin{align*}
    && \pi_t (C)=\frac{1}{(1+r)^{N-t}}\mathbb{E}^* [C \mid \mathcal{F}_t], && t=0,1,...,N
\end{align*}
where $\mathbb{P}^*$ denotes any risk-neutral probability measure.
\end{tcolorbox}

\subsection{(4.2) Pricing of Vanilla Options in the CRR Model}
\begin{tcolorbox}[enhanced, drop fuzzy shadow, title=Proposition 4.8]
The price $\pi_t(C)$ of the contingent claim $C=f(S_N^{(1)})$ satisfies
\begin{align*}
    && \pi_t(C) = v(t,S_t^{(1)}), && t=0,1,...,N
\end{align*}
where the function $v(t,x)$ is given by
\begin{align*}
    v(t,x) &= \frac{1}{(1+r)^{N-t}}\mathbb{E}^* \left[ f\left(x\prod^N_{j=t+1}(1+R_j) \right)\right]\\
    &= \frac{1}{(1+r)^{N-t}} \sum^{N-t}_{k=0} \binom{N-t}{k}(p^*)^k (1-p^*)^{N-t-k}f(x(1+b)^k (1+a)^{N-t-k})
\end{align*}
\end{tcolorbox}
\ \\
\begin{tcolorbox}[enhanced, drop fuzzy shadow, title=Useful note; Binomial Identity and Coefficient]
\begin{itemize}
    \item Binomial Identity
    \begin{align*}
    (x+y)^n = \sum_{k=0}^n \binom{n}{k} x^{n-k} y^k
    \end{align*}
    \item Binomial Coefficient
    \begin{align*}
    \binom{n}{k}=\frac{n!}{k!(n-k)!}
    \end{align*}
\end{itemize}
\end{tcolorbox}

\newpage
\section{(5.4) Ito Stochastic Integral: Ito Isometry}
\begin{tcolorbox}[enhanced, drop fuzzy shadow, title=Proposition 5.9; Ito Isometry]
\begin{align*}
    Var\left[\int^T_0 f(t)dB_t \right] = \mathbb{E}\left[\left(\int^T_0 f(t)dB_t \right)^2 \right] = \int^T_0 \big|f(t)\big|^2 dt
\end{align*}
\end{tcolorbox}

Application of Ito Isometry \\
\begin{itemize}
    \item i.e.1)\\
    \begin{align*}
    \mathbb{E}\left[\left(\int^T_0 B_t dB_t \right)^2 \right] = \mathbb{E}\left[\int^T_0 \big|B_t\big|^2 dt \right] = \int^T_0 \mathbb{E}\left[ \big|B_t\big|^2 \right]dt = \int^T_0 tdt = \frac{T^2}{2}
    \end{align*}
    \item i.e.2) For all square-integrable adapted processes $(u_t)_{t \in \mathbb{R}_+}$, $(v_t)_{t \in \mathbb{R}_+}$\\
    \begin{align*}
    \mathbb{E}\left[\int^T_0 u_t dB_t \int^T_0 v_t dB_t \right] = \mathbb{E}\left[ \int^T_0 u_t v_t dt \right]
    \end{align*}
\end{itemize}

\end{document}