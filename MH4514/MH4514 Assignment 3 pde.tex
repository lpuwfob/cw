% --------------------------------------------------------------
% This is all preamble stuff that you don't have to worry about.
% Head down to where it says "Start here"
% --------------------------------------------------------------
 
\documentclass[12pt]{article}
 
\usepackage[margin=1in]{geometry} 
\usepackage{mathtools,amsthm,amssymb,dsfont}
\usepackage{hyperref}
 
\newenvironment{theorem}[2][Theorem]{\begin{trivlist}
\item[\hskip \labelsep {\bfseries #1}\hskip \labelsep {\bfseries #2.}]}{\end{trivlist}}
\newenvironment{lemma}[2][Lemma]{\begin{trivlist}
\item[\hskip \labelsep {\bfseries #1}\hskip \labelsep {\bfseries #2.}]}{\end{trivlist}}
\newenvironment{exercise}[2][Exercise]{\begin{trivlist}
\item[\hskip \labelsep {\bfseries #1}\hskip \labelsep {\bfseries #2.}]}{\end{trivlist}}
\newenvironment{reflection}[2][Reflection]{\begin{trivlist}
\item[\hskip \labelsep {\bfseries #1}\hskip \labelsep {\bfseries #2.}]}{\end{trivlist}}
\newenvironment{proposition}[2][Proposition]{\begin{trivlist}
\item[\hskip \labelsep {\bfseries #1}\hskip \labelsep {\bfseries #2.}]}{\end{trivlist}}
\newenvironment{corollary}[2][Corollary]{\begin{trivlist}
\item[\hskip \labelsep {\bfseries #1}\hskip \labelsep {\bfseries #2.}]}{\end{trivlist}}
% Created
\newenvironment{solution}[2][Solution]{\begin{trivlist}
\item[\hskip \labelsep {\bfseries #1}\hskip \labelsep {\bfseries #2.}]}{\end{trivlist}}
\newenvironment{question}[2][Question]{\begin{trivlist}
\item[\hskip \labelsep {\bfseries #1}\hskip \labelsep {\bfseries #2.}]}{\end{trivlist}}

\begin{document}
 
% --------------------------------------------------------------
%                         Start here
% --------------------------------------------------------------
 
%\renewcommand{\qedsymbol}{\filledbox}
 
\title{Assignment 3 - MH4514 Financial Mathematics}
\author{Name: \ \ \ \ \ \ \  \underline{\texttt{Honda Naoki}}\\ 
Matriculation: \underline{\texttt{N1804369J}}} 

\maketitle

\begin{question}{1}
\end{question}
\begin{solution}[Solution] \\ 
\flushleft{Using the properties of covariance and Ito multiplication table, we can calculate the fraction as}
\begin{align*}
    \frac{Cov\left(\frac{dS_t}{S_t}, \frac{dM_t}{M_t}\right)}{Var\left[\frac{dM_t}{M_t}\right]} &= \frac{Cov\left(rdt+\beta \left(\frac{dM_t}{M_t}-rdt\right)+\sigma_S dW_t, \frac{dM_t}{M_t} \right)}{Var\left[\frac{dM_t}{M_t}\right]}\\
    &= \frac{Cov\left(\beta \frac{dM_t}{M_t} , \frac{dM_t}{M_t} \right)}{Var\left[\frac{dM_t}{M_t}\right]} = \beta
\end{align*}
\end{solution}

\ \\
\ \\
\begin{question}{2}
\end{question}
\begin{solution}[Solution] \\ 
\flushleft{By (1) and (2), observe that the evolution of $(S_t)_{t \in \mathbb{R}_+}$ is given by}
\begin{align*}
    \frac{dS_t}{S_t} &= rdt+ \beta \left(\frac{dM_t}{M_t}-rdt\right) +\sigma_S dW_t \\
    &= (r+\beta(\mu - r))dt+ \beta \sigma_M dB_t +\sigma_S dW_t
\end{align*}
\flushleft{Now, consider its volatility, by squaring both side we have}
\begin{align*}
    \left(\frac{dS_t}{S_t}\right)^2 &= \left((r+\beta(\mu - r))dt+ \beta \sigma_M dB_t +\sigma_S dW_t\right)^2\\
    &= \left( \beta^2 \sigma_M^2 +\sigma_S^2\right) dt
\end{align*}
\flushleft{Thus, we can think of $(S_t)_{t \in \mathbb{R}_+}$ as a geometric Brownian motion with volatility $\sqrt{\beta^2 \sigma_M^2 +\sigma_S^2}$. Finally with the excess return $\alpha$, we can express its evolution as}
\begin{align*}
    \frac{dS_t}{S_t} &= (r+\alpha+\beta(\mu -r))dt+ \sqrt{\beta^2 \sigma_M^2 +\sigma_S^2} dW_t \\
    dS_t &= (r+\alpha+\beta(\mu -r))S_t dt+ \sqrt{\beta^2 \sigma_M^2 +\sigma_S^2} S_t dW_t
\end{align*}
\end{solution}


\newpage
\begin{question}{3}
\end{question}
\begin{solution}[Solution] \\ 
\flushleft{Consider the equations (1), we have}
\begin{align*}
\frac{dM_t}{M_t} &= \mu dt + \sigma_M dB_t \\
&= rdt +\sigma_M \left( dB_t + \frac{\mu -r}{\sigma_M}dt \right)\\
&= rdt +\sigma_M dB_t^* \qquad \qquad \qquad \left(\text{where\ \ \ \ \ } dB_t^* = \left( dB_t + \frac{\mu -r}{\sigma_M}dt \right) \right)
\end{align*}
\flushleft{Next, we can rewrite (2) as}
\begin{align*}
    \frac{dS_t}{S_t} &= rdt+ \beta(M_t) \left(\frac{dM_t}{M_t}-rdt\right) +\sigma_S dW_t\\
    &= rdt+ \beta(M_t) (rdt +\sigma_M dB_t^*-rdt) +\sigma_S dW_t\\
    &= rdt+ \sigma_M \beta(M_t) dB_t^*+\sigma_S dW_t^*  \qquad \qquad \qquad \left(\text{where\ \ \ \ \ } dW_t^* = dW_t \right)
\end{align*}
\flushleft{Hence, by setting $B_t^* = \left( B_t + \frac{\mu -r}{\sigma_M}t \right)$ and $W_t^* = W_t$ we have }
\begin{align*}
    \begin{dcases}
    \frac{dM_t}{M_t} &= rdt +\sigma_M dB_t^*\\[10pt]
    \frac{dS_t}{S_t} &= rdt+ \sigma_M \beta(M_t) dB_t^*+\sigma_S dW_t^*
    \end{dcases}
\end{align*}
\end{solution}

\ \\
\ \\
\begin{question}{4}
\end{question}
\begin{solution}[Solution] \\ 
\flushleft{Consider the new probability measure $\mathbb{P}^*$ defined on $\mathcal{F}_T$ by}
\begin{align*}
    \frac{d\mathbb{P}^*}{d\mathbb{P}}= \text{exp}\bigg\{ -\frac{\mu-r}{\sigma_M}B_T -\frac{1}{2}\bigg( \frac{\mu-r}{\sigma_M} \bigg)^2 T \bigg\}
\end{align*}
\flushleft{Setting}
\begin{align*}
    B^*_t = B_t + \frac{\mu-r}{\sigma_M}t,\qquad W^*_t = W_t
\end{align*}
\flushleft{then under $\mathbb{P}^*$, the processes $(B^*_t)_{t \in \mathbb{R}_+}$ and $(W^*_t)_{t \in \mathbb{R}_+}$ are two independent standard Brownian motions.}
\end{solution}

\newpage
\begin{question}{5}
\end{question}
\begin{solution}[Solution] \\ 
\flushleft{From Question 4, we know under $\mathbb{P}^*$, the processes $(B^*_t)_{t \in \mathbb{R}_+}$ and $(W^*_t)_{t \in \mathbb{R}_+}$ are two independent standard Brownian motions. Using this, the discounted asset price processes $(\Tilde{S}_t, \Tilde{M}_t)_{t \in \mathbb{R}_+}$ can be shown as martingales.}
\begin{align*}
    \mathbb{E}^*\left[M_t \big|\mathcal{F}_k\right] &= e^{-rt}\mathbb{E}^*\left[\int^t_0 rM_u du + \sigma_M \int^t_0 M_u dB_u^* \bigg|\mathcal{F}_k\right]\\
    &= e^{-rt} \mathbb{E}^*\left[\int^t_0 rM_u du \bigg|\mathcal{F}_k\right] = e^{-rt} e^{(t-k)r}M_k\\
    &= \Tilde{M_k}
\end{align*}
\begin{align*}
    \mathbb{E}^*\left[\Tilde{S}_t \big|\mathcal{F}_k\right] &= e^{-rt}\mathbb{E}^*\left[\int^t_0 rS_u du + \sigma_M \int^t_0 \beta(M_u) S_u dB_u^* +\sigma_S \int^t_0 S_u dW_u^* \bigg|\mathcal{F}_k\right]\\
    &= e^{-rt}\mathbb{E}^*\left[\int^t_0 rS_u du \bigg|\mathcal{F}_k\right] = e^{-rt} e^{(t-k)r}S_k\\
    &= \Tilde{S_k}
\end{align*}
\flushleft{From Proposition 6.6, the probability measure $\mathbb{P}^*$ is risk-neutral since the discounted risky asset price processes $(\Tilde{S}_t, \Tilde{M}_t)_{t \in \mathbb{R}_+}$ are multingales under $\mathbb{P}^*$. \\
\ \\
Hence, by Theorem 6.8, the market based on the assets $S_t$ and $M_t$ is without arbitrage opportunities since it admits at least one equivalent risk-neutral probability measure $\mathbb{P}^*$.}
\end{solution}

\ \\
\begin{question}{6}
\end{question}
\begin{solution}[Solution] \\ 
\flushleft{By Ito's calculus we have}
\begin{align*}
    \begin{split}
        dV_t &= \xi_t dS_t + S_t d\xi_t + d\xi_t dS_t \\
        &\quad + \zeta_t dM_t + M_t d\zeta_t + d\zeta_t dM_t \\
        &\qquad + \eta_t dA_t + A_t d\eta_t + d\eta_t dA_t
    \end{split}\\
\end{align*}
\flushleft{Here, since we know $S_t d\xi_t + M_t d\zeta_t + A_t d\eta_t=0$ being true under the self-financing condition, and $d\xi_t dS_t = d\zeta_t dM_t = d\eta_t dA_t = 0$, we can rewrite $dV_t$ as}
\begin{align*}
    dV_t = \xi_t dS_t + \zeta_t dM_t + \eta_t dA_t
\end{align*}
\end{solution}

\newpage
\begin{question}{7}
\end{question}
\begin{solution}[Solution] \\ 
\flushleft{From Question 2 and the same derivation of Question 3 we have}
\begin{align*}
    \begin{dcases}
    dM_t &= rM_t dt +\sigma_M M_t dB_t^*\\[10pt]
    dS_t &= rS_t dt+ \sigma_M \beta(M_t)S_t dB_t^*+\sigma_S S_t dW_t^*
    \end{dcases}
\end{align*}
\flushleft{Using the above system and Ito's formula\footnote{In the text book, P191, (5.5.7)}, we have\footnote{Here we omit the argument bracket of function $f(t, S_t, M_t)$ for the sake of simplicity}}
\begin{align*}
    df(t,S_t, M_t) = \bigg\{\frac{\partial f}{\partial t}& + \frac{1}{2}\bigg(\sigma_M^2 \beta^2(M_t) +\sigma_S^2\bigg)S_t^2 \frac{\partial^2 f}{\partial x^2} +\frac{1}{2}\sigma_M^2 M_t^2 \frac{\partial^2 f}{\partial y^2} + \beta(M_t)\sigma_M^2 M_t S_t\frac{\partial^2 f}{\partial x \partial y}\bigg\}dt \\
    &+ \frac{\partial f}{\partial x}dS_t +\frac{\partial f}{\partial y}dM_t
\end{align*}
\flushleft{Further expand $dS_t$ and $dM_t$ we have}
\begin{align*}
    df(t,S_t, M_t) = \bigg\{\frac{\partial f}{\partial t} &+ \frac{1}{2}\bigg(\sigma_M^2 \beta^2(M_t) +\sigma_S^2\bigg)S_t^2 \frac{\partial^2 f}{\partial x^2} + \frac{1}{2}\sigma_M^2 M_t^2 \frac{\partial^2 f}{\partial y^2} \\
    &+ \beta(M_t)\sigma_M^2 M_t S_t\frac{\partial^2 f}{\partial x \partial y} + r\left(S_t\frac{\partial f}{\partial x} +M_t\frac{\partial f}{\partial y} \right)\bigg\}dt \\
    &\qquad +\bigg(\sigma_M \beta(M_t)S_t \frac{\partial f}{\partial x} + \sigma_M M_t\frac{\partial f}{\partial y} \bigg)dB_t^* +\sigma_S S_t \frac{\partial f}{\partial x}dW_t^*
    \tag{7.1}
\end{align*}
\flushleft{Using the self-financing condition of Question 6, we also can express $df(t,S_t, M_t)$ as follows}
\begin{align*}
    df(t,S_t, M_t) &= dV_t = \xi_t dS_t + \zeta_t dM_t + \eta_t dA_t \\
    &= \xi_t (rS_t dt+ \sigma_M \beta(M_t)S_t dB_t^*+\sigma_S S_t dW_t^*) + \zeta_t (rM_t dt +\sigma_M M_t dB_t^*) + r\eta_t A_t dt\\
    &= rf(t,S_t, M_t)dt + (\sigma_M \beta(M_t)\xi_t S_t + \zeta_t\sigma_M M_t )dB_t^* +\sigma_S \xi_t S_t dW_t^*
    \tag{7.2}
\end{align*}
\flushleft{By identification of the term in $dt$ in (7.1) and (7.2) we get the PDE with the terminal condition:}
\begin{align*}
\begin{dcases}
    rf(t,x, y) &= \frac{\partial f}{\partial t} + \frac{1}{2}\bigg(\sigma_M^2 \beta^2(M_t) +\sigma_S^2\bigg)x^2 \frac{\partial^2 f}{\partial x^2} +\frac{1}{2}\sigma_M^2 y^2 \frac{\partial^2 f}{\partial y^2}  \\
    &\quad+ \beta(M_t)\sigma_M^2 xy\frac{\partial^2 f}{\partial x \partial y} + r\left( x\frac{\partial f}{\partial x} +y\frac{\partial f}{\partial y} \right)\\[10pt]
    f(T,x,y)&=h(x, y)
\end{dcases}
\end{align*}
\end{solution}


\newpage
\begin{question}{8}
\end{question}
\begin{solution}[Solution] \\ 
\flushleft{Similar to the way we derived the PDE, by identifying the coefficient on $dB_t^*$ and $dB_t^*$ in (7.1)-(7.2) we have system of equation}
\begin{align*}
    \begin{dcases}
    \sigma_M \beta(M_t)S_t \frac{\partial f}{\partial x} + \sigma_M M_t\frac{\partial f}{\partial y} = \sigma_M \beta(M_t)\xi_t S_t + \zeta_t\sigma_M M_t \\[10pt]
    \sigma_S S_t \frac{\partial f}{\partial x} = \sigma_S \xi_t S_t 
    \end{dcases}
\end{align*}
\flushleft{which gives a solution}
\begin{align*}
    \xi_t = \frac{\partial f}{\partial x}(t,x,y), \qquad \zeta_t =\frac{\partial f}{\partial y}(t,x,y).
\end{align*}
\flushleft{For $\eta_t$, from previous result and the definition of the portfolio price $V_t$ we have}
\begin{align*}
    \eta_t A_t &= V_t - \xi_t S_t - \zeta_t M_t \\
    &= f(t,x,y) - S_t\frac{\partial f}{\partial x}(t,x,y) - M_t\frac{\partial f}{\partial y}(t,x,y)\\
    \eta_t &= \frac{1}{A_0 e^{rt}} \left\{f(t,x,y) - S_t\frac{\partial f}{\partial x}(t,x,y) - M_t\frac{\partial f}{\partial y}(t,x,y)\right\}
\end{align*}
\end{solution}

\newpage
\begin{question}{9}
\end{question}
\begin{solution}[Solution] \\ 
\flushleft{From Question 7, with constant $\beta$ and terminal condition of a call option payoff, we have PDE of}
\begin{align*}
\begin{dcases}
    rf(t,x, y) &= \frac{\partial f}{\partial t} + \frac{1}{2}\bigg(\sigma_M^2 \beta^2 +\sigma_S^2\bigg)x^2 \frac{\partial^2 f}{\partial x^2} +\frac{1}{2}\sigma_M^2 y^2 \frac{\partial^2 f}{\partial y^2}  \\
    &\quad+ \beta\sigma_M^2 xy\frac{\partial^2 f}{\partial x \partial y} + r\left( x\frac{\partial f}{\partial x} +y\frac{\partial f}{\partial y} \right)\\[10pt]
    f(T,x,y)&=(x-K)^+
\end{dcases}
\end{align*}

\flushleft{By change of variables with $u(\tau, a, b) = e^{-rt}f\bigg(T-t, e^{x+\left(r-(\sigma_M^2 \beta^2 + \sigma_S^2)/2\right)\tau}, e^{y+(r-\sigma_M^2/2)\tau}\bigg)$, \\
we have}
\begin{align*}
    \frac{\partial u}{\partial t} = \frac{\sigma_M^2 \beta^2 + \sigma_S^2}{2}\frac{\partial^2 u}{\partial a^2} + \frac{\sigma_M^2}{2}\frac{\partial^2 u}{\partial b^2} + \beta \sigma_M^2 \frac{\partial^2 u}{\partial a \partial b}
\end{align*}

\newpage
\textbf{\underline{Step 1}}
\flushleft{By change of variables with $v(t, x, y) = e^{-rt}f(t,x,y)$, such that $\frac{\partial v}{\partial t}=e^{-rt}\frac{\partial f}{\partial t}-re^{-rt}f$,\\
we have}
\begin{align*}
    \frac{\partial v}{\partial t} + \frac{\sigma_M^2 \beta^2 +\sigma_S^2}{2}x^2 \frac{\partial^2 v}{\partial x^2} +\frac{\sigma_M^2}{2}y^2 \frac{\partial^2 v}{\partial y^2} + \beta\sigma_M^2 xy\frac{\partial^2 v}{\partial x \partial y} + r\left( x\frac{\partial v}{\partial x} +y\frac{\partial v}{\partial y} \right) = 0
\end{align*}

\textbf{\underline{Step 2}}
\flushleft{By change of variables with $w(\tau, e^a, e^b) = v(T-t,x,y)$, such that}
\begin{align*}
\frac{\partial v}{\partial x} \to \frac{1}{x}\frac{\partial w}{\partial a},\quad
\frac{\partial^2 v}{\partial x^2} \to \frac{1}{x^2}\frac{\partial^2 w}{\partial a^2} - \frac{1}{x^2}\frac{\partial w}{\partial a},\quad
\frac{\partial v}{\partial t} \to -\frac{\partial w}{\partial \tau}
\end{align*}
\flushleft{we have}
\begin{align*}
    \frac{\partial w}{\partial \tau}
    -\frac{\sigma_M^2 \beta^2 +\sigma_S^2}{2} \frac{\partial^2 w}{\partial a^2}
    &-\frac{\sigma_M^2}{2} \frac{\partial^2 w}{\partial b^2}
    -\beta\sigma_M^2 \frac{\partial^2 w}{\partial a \partial b}\\
    &-\left(r-\frac{\sigma_M^2 \beta^2 +\sigma_S^2}{2}\right)\frac{\partial w}{\partial a}
    -\left(r-\frac{\sigma_M^2}{2}\right)\frac{\partial w}{\partial b}
    = 0
\end{align*}


\textbf{\underline{Step 3}}
\flushleft{By change of variables with $u(\tau, c, d) = w \left( \tau,x+\left(r-\frac{\sigma_M^2 \beta^2 +\sigma_S^2}{2}\right)\tau,y+\left(r-\frac{\sigma_M^2}{2}\right)\tau \right)$,\\ such that}
\begin{align*}
\frac{\partial w}{\partial \tau}
\to \frac{\partial u}{\partial \tau}
+\left(r-\frac{\sigma_M^2 \beta^2 +\sigma_S^2}{2}\right)\frac{\partial u}{\partial c}
+\left(r-\frac{\sigma_M^2}{2}\right)\frac{\partial u}{\partial d}, \qquad
\frac{\partial w}{\partial a} \to \frac{\partial w}{\partial c}
\end{align*}
\flushleft{we have}
\begin{align*}
    \frac{\partial u}{\partial \tau}
    -\frac{\sigma_M^2 \beta^2 +\sigma_S^2}{2} \frac{\partial^2 u}{\partial c^2}
    &-\frac{\sigma_M^2}{2} \frac{\partial^2 u}{\partial d^2}
    -\beta\sigma_M^2 \frac{\partial^2 u}{\partial c \partial d}
    = 0
\end{align*}

\href{https://quant.stackexchange.com/questions/38248/black-scholes-pde-change-of-variables}{Link 1}\\
\href{https://quant.stackexchange.com/questions/84/transformation-from-the-black-scholes-differential-equation-to-the-diffusion-equ}{Link 2}\\
\href{https://math.stackexchange.com/questions/2670875/change-of-variables-in-black-and-scholes-equation}{Link 3}\\
\href{http://www.math.tamu.edu/~stecher/425/Sp12/blackScholesHeatEquation.pdf}{Link 4}

\flushleft{Define function $g(t,x,y)$ as the 2 dimensional Gaussian probability density function}
\begin{align*}
    g(t,x,y) = \frac{1}{2\pi t}exp\bigg(-\frac{x^2+y^2}{2t}\bigg)
\end{align*}
\flushleft{which solves the heat equation with initial condition}
\begin{align*}
    g(0,x,y)=h(exp(\sigma x +\sigma y))
\end{align*}
\flushleft{i.e.}
\begin{align*}
        \frac{\partial g}{\partial t} = \bigg(-\frac{1}{t} + \frac{x^2 + y^2}{2t^2} \bigg)g(t,x,y) = \frac{1}{2}\bigg(\frac{\partial^2 g}{\partial x^2} + \frac{\partial^2 g}{\partial y^2}\bigg)
\end{align*}
\end{solution}

\begin{question}{10}
\end{question}
\begin{solution}[Solution] \\ 
\flushleft{\ }
\end{solution}


% --------------------------------------------------------------
% You don't have to mess with anything below this line.
% --------------------------------------------------------------
 
\end{document}