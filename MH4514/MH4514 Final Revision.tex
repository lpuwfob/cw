\documentclass[12pt]{extarticle}
\usepackage{tcolorbox}
\tcbuselibrary{skins} %preamble
\usepackage{tabularx}
%Some packages I commonly use.
\usepackage[english]{babel}
\usepackage{graphicx}
\usepackage{framed}
\usepackage[normalem]{ulem}
\usepackage{amsmath}
\usepackage{amsthm}
\usepackage{amssymb}
\usepackage{amsfonts}
\usepackage{enumerate}
\usepackage{mathtools}
\usepackage[utf8]{inputenc}
\usepackage[top=1 in,bottom=1in, left=1 in, right=1 in]{geometry}

%A bunch of definitions that make my life easier
\newcommand{\matlab}{{\sc Matlab} }
\newcommand{\cvec}[1]{{\mathbf #1}}
\newcommand{\rvec}[1]{\vec{\mathbf #1}}
\newcommand{\ihat}{\hat{\textbf{\i}}}
\newcommand{\jhat}{\hat{\textbf{\j}}}
\newcommand{\khat}{\hat{\textbf{k}}}
\newcommand{\minor}{{\rm minor}}
\newcommand{\trace}{{\rm trace}}
\newcommand{\spn}{{\rm Span}}
\newcommand{\rem}{{\rm rem}}
\newcommand{\ran}{{\rm range}}
\newcommand{\range}{{\rm range}}
\newcommand{\mdiv}{{\rm div}}
\newcommand{\proj}{{\rm proj}}
\newcommand{\R}{\mathbb{R}}
\newcommand{\N}{\mathbb{N}}
\newcommand{\Q}{\mathbb{Q}}
\newcommand{\Z}{\mathbb{Z}}
\newcommand{\<}{\langle}
\renewcommand{\>}{\rangle}
\renewcommand{\emptyset}{\varnothing}
\newcommand{\attn}[1]{\textbf{#1}}
\theoremstyle{definition}
\newtheorem{theorem}{Theorem}
\newtheorem{corollary}{Corollary}
\newtheorem*{definition}{Definition}
\newtheorem*{example}{Example}
\newtheorem*{note}{Note}
\newtheorem{exercise}{Exercise}
\newcommand{\bproof}{\bigskip {\bf Proof. }}
\newcommand{\eproof}{\hfill\qedsymbol}
\newcommand{\Disp}{\displaystyle}
\newcommand{\qe}{\hfill\(\bigtriangledown\)}
\setlength{\columnseprule}{1 pt}

\title{\textbf{MH4514 Financial Mathematics}\\
\Large - Final Revision -}
\author{Naoki Honda}
\date{May 2019}

\begin{document}

\maketitle
\section{Brownian Motion and Stochastic Calculus}
\begin{tcolorbox}[enhanced, drop fuzzy shadow, title=Definition 5.1]
The standard Brownian motion is a stochastic process $(B_t)_{t \in \mathbb{R}_+}$ such that
\begin{enumerate}
    \item $B_0 = 0$ almost surely,
    \item The sample trajectories $t \longrightarrow B_t$ are continuous, with probability 1.
    \item For any finite sequence of times $t_0 < t_1 < \cdots < t_n$, the increments
    \begin{align*}
        B_{t_1} - B_{t_0}, B_{t_2} - B_{t_1}, \cdots, B_{t_n} - B_{t_{n-1}}
    \end{align*}
    are mutually independent random variables.
    \item For any given times $0 \geq s < t$, $B_t -B_s$ has the Gaussian distribution $\mathcal{N}(0,t-s)$.
\end{enumerate}
\end{tcolorbox}

\newpage
\begin{tcolorbox}[enhanced, drop fuzzy shadow, title=Lemma 5.3]
The stochastic integral $\int^T_0 f(t)dB_t$ has the centered Gaussian distribution
\begin{align*}
    \int^T_0 f(t)dB_t \simeq \mathcal{N}\bigg( 0, \int^T_0 |f(t)|^2 dt \bigg)
\end{align*}
\end{tcolorbox}

\begin{tcolorbox}[enhanced, drop fuzzy shadow, title=Stochastic Calculus]
\begin{align*}
    df(t,X_t) = \frac{\partial f}{\partial t}(t,X_t)dt + \frac{\partial f}{\partial x}(t,X_t)dX_t + \frac{1}{2}\frac{\partial^2 f}{\partial x^2}(t,X_t)(dX_t)^2
\end{align*}
\end{tcolorbox}

\begin{tcolorbox}[enhanced, drop fuzzy shadow, title=Proposition 5.12]
Given the stochastic differential equation
\begin{align*}
    dS_t = \mu S_t dt +\sigma S_t dB_t
\end{align*}
The solution is given by
\begin{align*}
    S_t = S_0 \text{ exp}\bigg\{\bigg(\mu - \frac{\sigma^2}{2} \bigg) t + \sigma B_t \bigg\}
\end{align*}
\end{tcolorbox}

\newpage
\section{The Black-Scholes PDE}
\begin{tcolorbox}[enhanced, drop fuzzy shadow, title=Proposition 6.1]
A portfolio allocation $(\xi_t, \eta_t)_{t\in \mathbb{R}_+}$ with price
\begin{align*}
    V_t = \eta_t A_t + \xi_t S_t
\end{align*}
is self-financing if and only if the relation
\begin{align*}
    dV_t = \eta_t dA_t + \xi_t dS_t
\end{align*}
holds.
\end{tcolorbox}

\begin{tcolorbox}[enhanced, drop fuzzy shadow, title=Definition 6.4]
A probability measure $\mathbb{P}^*$ on $\Omega$ is called a risk-neutral measure if it satisfies
\begin{align*}
    \mathbb{E}^*[S_t|\mathcal{F}_u] = e^{(t-u)r}S_u, \qquad 0\leq u \leq t,
\end{align*}
where $\mathbb{E}^*$ denotes the expectation under $\mathbb{P}^*$.
\end{tcolorbox}

\begin{tcolorbox}[enhanced, drop fuzzy shadow, title=Proposition 6.13]
The arbitrage price $\pi_t (C)$ at time $t \in [0,T]$ of the option with payoff $C=h(S_T)$ is given by $\pi_t (C) = g(t,S_t)$, where the function $g(t,x)$ is solution of the following Black Scholes PDE:
\begin{align*}
    \begin{dcases}
    rg(t,x) = \frac{\partial g}{\partial t}(t,x) + rx\frac{\partial g}{\partial x}(t,x) + \frac{1}{2}\sigma^2 x^2 \frac{\partial^2 g}{\partial x^2}(t,x),\\
    g(T,x)=h(x), \qquad x>0.
    \end{dcases}
\end{align*}
\end{tcolorbox}

\begin{tcolorbox}[enhanced, drop fuzzy shadow, title=Proposition 6.17]
When $h(x)=(x-K)^+$, the solution of the Black-Scholes PDE is given by
\begin{align*}
    f(t,x)=x\Phi(d_+ (T-t))-Ke^{-(T-t)r}\Phi(d_- (T-t)),
\end{align*}
where
\begin{align*}
    \Phi(x)=\frac{1}{\sqrt{2\pi}} \int^x_{-\infty}e^{-y^2/2}dy, \qquad x\in \mathbb{R},
\end{align*}
and
\begin{align*}
    d_\pm (T-t) := \frac{\text{log}(x/K) + (r\pm \sigma^2/2)(T-t)}{|\sigma|\sqrt{T-t}}
\end{align*}
\end{tcolorbox}

\newpage
\section{Martingale Approach to Pricing and Hedging}
\subsection{Prep}
\begin{tcolorbox}[enhanced, drop fuzzy shadow, title=Proposition 7.1]
The indefinite stochastic integral $\Big(\int^t_0 u_s dB_s\Big)_{t\in \mathbb{R}_+}$ of a square-integrable adapted process $u\in L^2_{ad}(\Omega \times \mathbb{R}_+)$ is a martingale, \textit{i.e.}:
\begin{align*}
    \mathbb{E}\bigg[\int^t_0 u_\tau dB_\tau \bigg|\mathcal{F}_s \bigg] = \int^s_0 u_\tau dB_\tau, \qquad 0\leq s \leq t
\end{align*}
\end{tcolorbox}

\begin{tcolorbox}[enhanced, drop fuzzy shadow, title=Theorem 7.2; Girsanov Theorem]
Let $(\phi_t)_{t\in [0,T]}$ be an adapted process satisfying the Novikov integrability condition
\begin{align*}
    \mathbb{E}\bigg[\text{exp}\bigg( \frac{1}{2} \int^T_0 |\psi_t|^2 dt\bigg) \bigg] < \infty,
\end{align*}
and let $\mathbb{Q}$ denote the probability measure defined by
\begin{align*}
    \frac{d\mathbb{Q}}{d\mathbb{P}} = \text{exp}\bigg(-\int^T_0 \psi_s dB_s -\frac{1}{2}\int^T_0 |\psi_s|^2 ds \bigg)
\end{align*}
Then
\begin{align*}
    \Hat{B}_t := B_t + \int^t_0 \psi_s ds, \qquad 0\leq t \leq T
\end{align*}
\end{tcolorbox}

\newpage
\subsection{Pricing by the Martingale Method}
Recall that from the first fundamental theorem of mathematical finance, a continuous market is without arbitrate opportunities if there exist (at least) a risk-neutral probability measure $\mathbb{P}^*$ under which the discounted price process
\begin{align*}
    \Tilde{S}_t := e^{-rt}S_t, \qquad t \in \mathbb{R}_+
\end{align*}
is a martingale under $\mathbb{P}^*$. In addition, when the risk-neutral probability measure is unique, the market is said to be \textit{complete}.\\
In case the price process $(S_t)_{t \in \mathbb{R}_+}$ satisfies the equation
\begin{align*}
    \frac{dS_t}{S_t} = \mu dt + \sigma dB_t, \qquad t\in \mathbb{R}_+, \qquad S_0 > 0
\end{align*}
we have
\begin{align*}
    S_t = S_0 e^{(\mu -\sigma^2/2)t +\sigma B_t}, \qquad t\in \mathbb{R}_+
\end{align*}
and the discounted price process
\begin{align*}
    \Tilde{S}_t := e^{-rt}S_t = S_0 e^{(\mu -r -\sigma^2/2)t +\sigma B_t}, \qquad t\in \mathbb{R}_+
\end{align*}
is a martingale under the probability measure $\mathbb{P}^*$ defined by
\begin{align*}
    \frac{d\mathbb{P}^*}{d\mathbb{P}} = \exp \bigg( -\frac{\mu -r}{\sigma} B_T -\frac{(\mu -r)^2}{2\sigma^2}T \bigg)
\end{align*}
which makes $\hat{B}_t := \frac{\mu -r}{\sigma}t +B_t$ a standard Brownian motion.\\
Moreover, we have
\begin{align*}
    d\Tilde{S}_t &= (\mu -r) \Tilde{S}_t dt +\sigma \Tilde{S}_t dB_t\\
    &= \sigma \Tilde{S}_t\bigg(\frac{\mu -r}{\sigma}dt +dB_t \bigg)\\
    &= \sigma \Tilde{S}_t d\hat{B}_t, \qquad t \in \mathbb{R}_+
\end{align*}
hence the discounted value $\Tilde{V}_t$ of a self-financing portfolio can be written as
\begin{align*}
    \Tilde{V}_t &= \Tilde{V}_0 +\int^t_0 \xi_u d\Tilde{S}_u\\
    &= \Tilde{V}_0 +\sigma \int^t_0 \xi_u \Tilde{S}_u d\hat{B}_u
\end{align*}
and by Proposition 7.1 it becomes a martingale under $\mathbb{P}^*$

\begin{tcolorbox}[enhanced, drop fuzzy shadow, title=Theorem 7.3]
Let $(\xi_t, \eta_t)_{t \in [0,T]}$ be a portfolio strategy with price
\begin{align*}
    V_t = \eta A_t + \xi_t S_t, \qquad t \in [0,T]
\end{align*}
and let $C$ be a contingent claim, such that
\begin{enumerate}
    \item $(\xi_t, \eta_t)_{t \in [0,T]}$ is a self-financing portfolio
    \item $(\xi_t, \eta_t)_{t \in [0,T]}$ hedges the claim $C$, \textit{i.e.} we have $V_T = C$
\end{enumerate}
Then the arbitrage price of the claim $C$ is given by
\begin{align*}
    \pi_t (C) = V_t = e^{-(T-t)r} \mathbb{E}^*[C|\mathcal{F}_t], \qquad 0 \leq t \leq T
\end{align*}
where $\mathbb{E}^*$ denotes expectation under the risk-neutral probability measure $\mathbb{P}^*$.
\end{tcolorbox}

\underline{\textbf{Proof:}}\\
Since the portfolio strategy $(\xi_t, \eta_t)_{t \in \mathbb{R}_+}$ is self-financing, we have
\begin{align*}
    \Tilde{
    V}_t = \Tilde{
    V}_0 +\sigma \int^t_0 \xi_u \Tilde{
    S}_u d\hat{B}_u = \Tilde{
    V}_0 +\int^t_0 \xi_u d\hat{S}_u, \qquad t \in \mathbb{R}_+
\end{align*}
which is a martingale under $\mathbb{P}^*$ from Proposition 7.1, hence
\begin{align*}
    \Tilde{V}_t &= \mathbb{E}^*[\Tilde{V}_T |\mathcal{F}_t]\\
    &= e^{-rT}\mathbb{E}^*[V_T |\mathcal{F}_t]\\
    &= e^{-rT}\mathbb{E}^*[C |\mathcal{F}_t]
\end{align*}
which implies
\begin{align*}
    V_t = e^{rt} \Tilde{V}_t = r^{-(T-t)r} \mathbb{E}^* [C|\mathcal{F}_t], \qquad 0 \leq t \leq T
\end{align*}

\begin{tcolorbox}[enhanced, drop fuzzy shadow, title=Proposition 7.4]
The price at time $t$ of a European call option with strike price $K$ and maturity $T$ is given by
\begin{align*}
    C(t, S_t) = S_t \Phi\big(d_+(T-t)\big) -Ke^{-(T-t)r} \Phi\big(d_-(T-t)\big), \qquad 0 \leq t \leq T
\end{align*}
with
\begin{align*}
    d_\pm (T-t):= \frac{\log(x/K) +(r \pm \sigma^2/2)(T-t)}{\sigma \sqrt{T-t}}
\end{align*}
\end{tcolorbox}

\underline{\textbf{Proof:}}\\
Using the relation
\begin{align*}
    S_T = S_t e^{(r-\sigma^2/2)(T-t) +\sigma(\hat{B}_T -\hat{B}_t)}, \qquad 0\leq t \leq T
\end{align*}
by Theorem 7.3 the price of the portfolio hedging $C$ is given by
\begin{align*}
    V_t &= e^{-(T-t)r}\mathbb{E}^*[C|\mathcal{F}_t]\\
    &= e^{-(T-t)r}\mathbb{E}^*[(S_T -K)^+|\mathcal{F}_t]\\
    &= e^{-(T-t)r}\mathbb{E}^*[(S_t e^{(r-\sigma^2/2)(T-t) +\sigma(\hat{B}_T -\hat{B}_t)} -K)^+|\mathcal{F}_t] \\
    &= e^{-(T-t)r}\mathbb{E}^*[(x e^{(r-\sigma^2/2)(T-t) +\sigma(\hat{B}_T -\hat{B}_t)} -K)^+]_{x=S_t}\\
    &= e^{-(T-t)r}\mathbb{E}^*[(e^{m(x) +X} -K)^+]_{x=S_t}
\end{align*}
where 
\begin{align*}
    m(x):= (T-t)r -\frac{\sigma^2}{2}(T-t) +\log x
\end{align*}
and
\begin{align*}
    X := \sigma(\hat{B}_T -\hat{B}_t) \simeq \mathcal{N}(0,\sigma^2 (T-t))
\end{align*}
is a centered Gaussian random variable with variance
\begin{align*}
    \text{Var}[X] = \text{Var}[\sigma(\hat{B}_T -\hat{B}_t)] = \sigma^2\text{Var}[\hat{B}_T -\hat{B}_t] = \sigma^2 (T-t)
\end{align*}
under $\mathbb{P}^*$. Hence we have
\begin{align*}
    C(t,S_t) &= V_t\\
    &= e^{-(T-t)r}\mathbb{E}^*[(e^{m(x) +X} -K)^+]_{x=S_t}\\
    &= e^{-(T-t)r} e^{m(S_t)+\sigma^2(T-t)/2}\Phi\bigg(v + \frac{m(S_t) -\log K}{v}\bigg)\\
    & \quad -Ke^{-(T-t)r}\Phi\bigg(\frac{m(S_t) -\log K}{v}\bigg)\\
    &= S_t\Phi\bigg(v + \frac{m(S_t) -\log K}{v}\bigg) -Ke^{-(T-t)r}\Phi\bigg(\frac{m(S_t) -\log K}{v}\bigg)\\
    &= S_t \Phi\big(d_+(T-t)\big) -Ke^{-(T-t)r} \Phi\big(d_-(T-t)\big), \qquad 0 \leq t \leq T
\end{align*}





\end{document}