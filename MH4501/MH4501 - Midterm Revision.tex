\documentclass[12pt]{extarticle}
%Some packages I commonly use.
\usepackage[english]{babel}
\usepackage{graphicx}
\usepackage{framed}
\usepackage[normalem]{ulem}
\usepackage{amsmath}
\usepackage{amsthm}
\usepackage{amssymb}
\usepackage{amsfonts}
\usepackage{enumerate}
\usepackage[utf8]{inputenc}
\usepackage[top=1 in,bottom=1in, left=1 in, right=1 in]{geometry}

%A bunch of definitions that make my life easier
\newcommand{\matlab}{{\sc Matlab} }
\newcommand{\cvec}[1]{{\mathbf #1}}
\newcommand{\rvec}[1]{\vec{\mathbf #1}}
\newcommand{\ihat}{\hat{\textbf{\i}}}
\newcommand{\jhat}{\hat{\textbf{\j}}}
\newcommand{\khat}{\hat{\textbf{k}}}
\newcommand{\minor}{{\rm minor}}
\newcommand{\trace}{{\rm trace}}
\newcommand{\spn}{{\rm Span}}
\newcommand{\rem}{{\rm rem}}
\newcommand{\ran}{{\rm range}}
\newcommand{\range}{{\rm range}}
\newcommand{\mdiv}{{\rm div}}
\newcommand{\proj}{{\rm proj}}
\newcommand{\R}{\mathbb{R}}
\newcommand{\N}{\mathbb{N}}
\newcommand{\Q}{\mathbb{Q}}
\newcommand{\Z}{\mathbb{Z}}
\newcommand{\<}{\langle}
\renewcommand{\>}{\rangle}
\renewcommand{\emptyset}{\varnothing}
\newcommand{\attn}[1]{\textbf{#1}}
\theoremstyle{definition}
\newtheorem{theorem}{Theorem}
\newtheorem{corollary}{Corollary}
\newtheorem*{definition}{Definition}
\newtheorem*{example}{Example}
\newtheorem*{note}{Note}
\newtheorem{exercise}{Exercise}
\newcommand{\bproof}{\bigskip {\bf Proof. }}
\newcommand{\eproof}{\hfill\qedsymbol}
\newcommand{\Disp}{\displaystyle}
\newcommand{\qe}{\hfill\(\bigtriangledown\)}
\setlength{\columnseprule}{1 pt}


\title{\textbf{MH4501 Multivariate Analysis}\\
\Large - Midterm Revision -}
\author{Naoki Honda}
\date{March 2019}

\begin{document}

\maketitle

\section{Multivariate Population and Sample Statistics}
\subsection{Population Covariance Matrix}
\begin{eqnarray*}
{\bf \Sigma} = Cov(X) &=& {\bf E}[(X-\mu)(X-\mu)^T]\\
&=& \left(
    \begin{array}{cccc}
      \sigma_{11} & \sigma_{12} & \ldots & \sigma_{1p} \\
      \sigma_{21} & \sigma_{22} & \ldots & \sigma_{2p} \\
      \vdots & \vdots & \ddots & \vdots \\
      \sigma_{p1} & \sigma_{p2} & \ldots & \sigma_{pp}
    \end{array} \right)
\end{eqnarray*} 

\subsection{Population Correlation Matrix}
\begin{eqnarray*}
\rho_{jk} = cor(X_j, X_k) = \frac{\sigma_{jk}}{\sqrt{\sigma_{jj}}\sqrt{\sigma_{kk}}}
\end{eqnarray*}
\begin{eqnarray*}
\rho = \left(
    \begin{array}{cccc}
      1 & \rho_{12} & \ldots & \rho_{1p} \\
      \rho_{21} & 1 & \ldots & \rho_{2p} \\
      \vdots & \vdots & \ddots & \vdots \\
      \rho_{p1} & \rho_{p2} & \ldots & 1
    \end{array} \right)
\end{eqnarray*}

\newpage
\subsection{Population Covariance Matrix and Correlation Matrix}
Let
\begin{eqnarray*}
V = diag(\sigma_{11}, \sigma_{22}, ... , \sigma_{pp}) = \left(
    \begin{array}{cccc}
      \sigma_{11} & 0 & \ldots & 0 \\
      0 & \sigma_{22} & \ldots & 0 \\
      \vdots & \vdots & \ddots & \vdots \\
      0 & 0 & \ldots & \sigma_{pp}
    \end{array} \right) 
\end{eqnarray*}

\begin{eqnarray*}
V^{1/2} = diag(\sqrt{\sigma_{11}}, \sqrt{\sigma_{22}}, ..., \sqrt{\sigma_{pp}})\\
\end{eqnarray*}

\begin{eqnarray*}
V^{-1/2} = (V^{1/2})^{-1}
\end{eqnarray*}

We have: 
\begin{eqnarray*}
{\bf \Sigma} &=& V^{1/2}\rho V^{1/2}\\
\rho &=& V^{-1/2} {\bf \Sigma} \  V^{-1/2}
\end{eqnarray*}

\subsection{Linear Transformation}
\flushleft{
When another r.v. $Y$ is defined as $Y_{q \times 1} := AX + b$, \\
we have:
}
\begin{eqnarray*}
\mu_Y = A\mu_X + b \\
{\bf \Sigma}_Y = A {\bf \Sigma}_X A^T
\end{eqnarray*}

\subsection{Sample Mean Vector}
\begin{eqnarray*}
\Bar{x} = \frac{1}{n}
    \left(\begin{array}{cccc}
      x_{11} & x_{21} & \ldots & x_{n1} \\
      x_{12} & x_{22} & \ldots & x_{n2} \\
      \vdots & \vdots & \ddots & \vdots \\
      x_{1p} & x_{2p} & \ldots & x_{np}
    \end{array} \right)
    \left(\begin{array}{c}
      1 \\ 1 \\ \vdots \\ 1
    \end{array} \right)
=\frac{1}{n} X^T {\bf 1}_{n \times 1}
\end{eqnarray*}

\subsection{Sample Covariance Matrix}
\begin{eqnarray*}
SSCP = (n-1)S &=& \frac{1}{n}
    \left(\begin{array}{ccc}
      \sum^n_{i=1}(x_{i1}-\Bar{x}_1)^2 & \ldots & \sum^n_{i=1}(x_{i1}-\Bar{x}_1)(x_{ip}-\Bar{x}_p) \\
      \vdots & \ddots & \vdots \\
      \sum^n_{i=1}(x_{ip}-\Bar{x}_p)(x_{i1}-\Bar{x}_1) & \ldots & \sum^n_{i=1}(x_{ip}-\Bar{x}_p)^2
    \end{array} \right) \\
&=& X^T (I_n - \frac{1}{n} {\bf 11}^T) X \\
&=& X^T X -n \Bar{x} \Bar{x}^T
\end{eqnarray*}

\newpage
\section{Multivariate Normal Distribution: MVN}
\subsection{PDF of MVN $N_p(\mu, \Sigma)$}
\begin{eqnarray*}
f(x) = \frac{1}{(2\pi)^{p/2} det(\Sigma)^{1/2}} \exp\left(-\frac{1}{2} (x-\mu)^T \Sigma^{-1} (x-\mu)\right)
\end{eqnarray*}

\subsection{Properties of MVN}
\subsubsection{Linear Transformation} 
Linear transformation gives a new MVN
\begin{eqnarray*}
AX+b \sim N_q(A\mu + b, A\Sigma A^T)
\end{eqnarray*}

\subsubsection{Partition and independence}
In the situation where
\begin{eqnarray*}
X=
\left( \begin{tabular}{l} 
        $X^{(1)}_{p \times 1}$\\ 
        $X^{(2)}_{q \times 1}$
        \end{tabular}\right)
\sim N_{p+q} \left(
        \mu = \left( \begin{tabular}{l} 
            $\mu^{(1)}_{p \times 1}$\\ 
            $\mu^{(2)}_{q \times 1}$
        \end{tabular}\right),
        \Sigma = \left(\begin{array}{cc}
            \Sigma^{(11)}_{p \times p} & \Sigma^{(12)}_{p \times q} \\
            \Sigma^{(21)}_{q \times p} & \Sigma^{(22)}_{q \times q}
        \end{array} \right)
    \right)
\end{eqnarray*}
We have: \\
\begin{eqnarray*}
\text{$X^{(1)}$ and $X^{(2)}$ are independent} \Leftrightarrow cov(X^{(1)},X^{(2)}) = \Sigma^{(12)} = {\bf 0}_{p \times q}
\end{eqnarray*}

\subsection{MVN Sampling Distribution}
\subsubsection{Distribution of Mean Vector}
$\Bar{x} \sim N_p\left(\mu, \frac{1}{n}\Sigma \right)$. Or equivalently, $\sqrt{n}(\Bar{x}-\mu) \sim N_p \left(0_p, \Sigma \right)$

\subsubsection{Whishart Distribution}
$SSCP = (n-1)S \sim W_p(n-1, \Sigma)$, the \underline{Wishart distribution}.
\begin{itemize}
    \item When $p=1$ and $\Sigma_{1 \times 1}=1$, $W_1(n,1)$ reduces to the $\chi^2 (n)$ \\ \ \\
    \item If $W_1 \sim W_p(n_1, \Sigma)$, $W_2 \sim W_p(n_2, \Sigma)$, and $W_1$ and $W_2$ are independent, \\
    then $W_1 + W_2 \sim W_p(n_1 + n_2, \Sigma)$ \\ \ \\
        \item If $W \sim W_p(n,\Sigma)$, and $B_{q \times p}$ is a given matrix, \\
    then $BWB^T \sim W_p (n, B\Sigma B^T)$
\end{itemize}

\subsubsection{Hotelling's $T^2$}
Let $x_1, x_2, ..., x_n$ be $n$ independent realizations from $N_p(\mu, \Sigma)$, then
\begin{itemize}
    \item $n(\Bar{x}-\mu)^T \Sigma^{-1} (\Bar{x}-\mu) \sim \chi^2 (p)$ \\
    \item $T^2 := n(\Bar{x}-\mu)^T S^{-1} (\Bar{x}-\mu)$ is called \textbf{Hotelling's $T^2$ statistic} \\
    \item $\frac{n-p}{p(n-1)} T^2 \sim F(p, n-p)$ \\
    \item When $n$ is large (relative to $q$),
    \begin{itemize}
        \item $T^2 \sim \chi^2 (p)$ \\
        \item $d_i^2 = (x_i - \Bar{x})^T S^{-1} (x_i - \Bar{x}) \sim \chi^2 (p)$
    \end{itemize}
\end{itemize}

\newpage
\section{Multivariate Statistical Inference}
\subsection{Confidence Region of Mean Vector}
A region $\mathcal{R} \subset \mathbb{R}^p$ is said to be a confidence region of $\mu_{p \times 1}$ \\
at confidence level $1-\alpha$, if $Pr[\mathcal{R} \owns \mu] = 1-\alpha$ \\
\ \\ 
Since $T^2 = n(\Bar{x}-\mu)^T S^{-1} (\Bar{x}-\mu)$ and $\frac{n-p}{p(n-1)}T^2 \sim F(p, n-p)$, we have
\begin{eqnarray*}
\mathcal{R} = \left\{ \mu : (\Bar{x}-\mu)^T S^{-1} (\Bar{x}-\mu) < \frac{p(n-1)}{n(n-p)}F_{\alpha}[p, n-p] \right\}
\end{eqnarray*}

\subsection{Simultaneous Confidence Interval: Bonferroni method}
\begin{eqnarray*}
\mathcal{R}_j = \left( \Bar{x}_j - t_{\alpha^*/2}[n-1] \sqrt{\frac{s_{jj}}{n}}, \Bar{x}_j + t_{\alpha^*/2}[n-1] \sqrt{\frac{s_{jj}}{n}} \right)
\end{eqnarray*}
where $\alpha^* = \alpha/p$. \\

\subsection{Situation when $n$ is large}
\begin{itemize}
    \item \underline{CR}:
    $\mathcal{R} = \left\{ \mu : (\Bar{x}-\mu)^T S^{-1} (\Bar{x}-\mu) < \frac{1}{n}\chi^2_{\alpha}[p] \right\}$
    \item \underline{SCI}:
    $\mathcal{R}_j = \left( \Bar{x}_j - z_{\alpha^*/2} \sqrt{\frac{s_{jj}}{n}}, \Bar{x}_j + z_{\alpha^*/2} \sqrt{\frac{s_{jj}}{n}} \right)$
\end{itemize}

\subsection{Two-Sample Comparison: \underline{Paired} Observations}
Consider 
\begin{eqnarray*}
d_1 = x_{11}-x{21}, d_2 = x_{12}-x{22}, ..., d_n = x_{1n}-x{2n} 
\end{eqnarray*}
And test $H_0: \mu_d = {\bf 0}_{p \times 1}$ \\
If we assume $d_i$ are from MVN, same as one-sample, we have
\begin{eqnarray*}
T^2 = n\Bar{d}^T S_d^{-1} \Bar{d} \end{eqnarray*}
\begin{eqnarray*}
\frac{n-p}{p(n-1)}T^2 \sim F(p, n-p)
\end{eqnarray*}
The arguments for CR and SCI are also same as one for the one-sample, changing $\Bar{x} \rightarrow \Bar{d}$ 

\newpage
\subsection{Two-Sample Comparison: \underline{Unpaired} Observations}
\underline{Assumptions}
\begin{itemize}
    \item Both sample 1 and 2 are independent\\
    \item Both Population 1 and 2 are MVN\\
    \item Two populations have the same population covariance matrix: $\Sigma_1 = \Sigma_2 = \Sigma$
\end{itemize}

We use \textbf{Pooled Sample Covariance Matrix}:
\begin{eqnarray*}
S_{pool} = \frac{(n_1 -1)S_1 + (n_2 -1)S_2}{n_1 + n_2 -2} = \frac{SSCP_1 + SSCP_2}{n_1 + n_2 -2}
\end{eqnarray*}
to estimate the unknown $\Sigma$\\

\subsubsection{Hotteling's $T^2$ statistic}
\begin{eqnarray*}
\Bar{x}_1 - \Bar{x}_2 \sim N_p \left( \mu_1 - \mu_2, \left( \frac{1}{n_1} + \frac{1}{n_2} \right) \Sigma \right)
\end{eqnarray*}
Thus
\begin{eqnarray*}
[(\Bar{x}_1 - \Bar{x}_2) - \mu_1 - \mu_2]^T \left( \frac{1}{n_1} + \frac{1}{n_2} \right)^{-1} \Sigma^{-1} [(\Bar{x}_1 - \Bar{x}_2) - \mu_1 - \mu_2] \sim \chi^2(p)
\end{eqnarray*}
and
\begin{eqnarray*}
T^2 = [(\Bar{x}_1 - \Bar{x}_2) - \mu_1 - \mu_2]^T \left( \frac{1}{n_1} + \frac{1}{n_2} \right)^{-1} S_{pool}^{-1} [(\Bar{x}_1 - \Bar{x}_2) - \mu_1 - \mu_2]
\end{eqnarray*}
with
\begin{eqnarray*}
\frac{n_1 + n_2 - 1 - p}{p(n_1 + n_2 -2)}
T^2 \sim F(p, n_1 + n_2 -1-p)
\end{eqnarray*}

\newpage
\subsubsection{Hypothesis Testing under $H_0: \mu_1 = \mu_2$}
We reject $H_0$ if the observed value of $\frac{n_1 + n_2 -1-p}{p(n_1 + n_2 -2)}T^2$ is greater than critical value $F_{\alpha} [p, n_1 + n_2 -1-p]$, where $\alpha$ is the desired significant level.\\

\subsubsection{CR of $\delta = \mu_1 - \mu_2$ at confidence level $1-\alpha$}
\begin{eqnarray*}
\mathcal{R} = \left\{ \delta : (\Bar{x}_1 - \Bar{x}_2 -\delta)^T \left( \frac{1}{n_1}+\frac{1}{n_2} \right)^{-1} S^{-1}_{pool} (\Bar{x}_1 - \Bar{x}_2 -\delta) < \frac{p(n_1 + n_2 -2)}{n_1 + n_2 -1-p} F_{\alpha}[p, n_1 + n_2 -1-p] \right\}
\end{eqnarray*}

\subsubsection{Bonferroni SCIs of $\mu_d$ at confidence level $1-\alpha$}
\begin{eqnarray*}
\mathcal{R}_j = \Bigg( (\Bar{x}_1 - \Bar{x}_2)_j - t_{\alpha^*/2}[n_1 + n_2 -2] \sqrt{\left( \frac{1}{n_1}+\frac{1}{n_2} \right)^{-1} s_{pool,jj}}, \\
(\Bar{x}_1 - \Bar{x}_2)_j + t_{\alpha^*/2}[n_1 + n_2 -2] \sqrt{\left( \frac{1}{n_1}+\frac{1}{n_2} \right)^{-1} s_{pool,jj}} \Bigg)
\end{eqnarray*}
where $\alpha^* = \alpha/p$


\section{MANOVA: Multivariate Analysis of Variance}
\subsection{Step 1}
We want to test the null hypothesis $H_0: \mu_1 = \mu_2 = ...  = \mu_G$ \\
Decompose population mean vector of each population as \\
\begin{eqnarray*}
\mu_g = \mu + (\mu_g -\mu) = \mu + \tau_g
\end{eqnarray*}
where $\tau_g = \mu_g - \mu$ is the treatment effect of Population $g$. \\
Therefore, it is equivalent to test $H_0: \tau_1 = \tau_2 =...= \tau_g = {\bf 0}_{p \times 1}$ \\


\subsection{Step 2}
Furthermore, let $X_g$ be any realization/observation of Population $g$, then we can decompose $X_g$ as \\
\begin{eqnarray*}
X_g = \mu + \tau_g + (X_g - \mu_g) = \mu + \tau_g + e_g
\end{eqnarray*}
where $e_g = X_g - \mu_g \sim N_p({\bf 0}_{p \times 1}, \Sigma)$ is the term of random error.


\subsection{Step 3}
Now assume we have a sample from each of the populations: \\
\underline{\bf For Sample $g$, we have $n_g$ realizations from Population g: $N_p (\mu_g, \Sigma)$} \\
\ \\ 
Following the discussion on the precious page, we decompose every $x_{gi}$ as \\
\begin{eqnarray*}
x_{gi} = \underbrace{\Bar{x}}_\text{overall sample mean vector} + \underbrace{(\Bar{x}_g - \Bar{x})}_\text{estimated treatment effect} + \underbrace{(x_{gi} - \Bar{x}_g)}_\text{residual}
\end{eqnarray*}


\subsection{Step 4}
Following the decomposition of $x_gi$, we have $x_{gi} - \Bar{x} = (\Bar{x}_g -\Bar{x}) + (x_{gi} -\Bar{x}_g)$.\\
Therefore \footnote{Noticing that $\sum_{i=1}^{n_g} (\Bar{x}_g -\Bar{x})(x_{gi} -\Bar{x}_g)^T = {\bf 0}_{p \times p}$ and $\sum_{i=1}^{n_g} (x_{gi} -\Bar{x}_g)(\Bar{x}_g -\Bar{x})^T = {\bf 0}_{p \times p}$.\\  (The sums of crossing terms are zero matrix.)},
\begin{eqnarray*}
\sum_{i=1}^{n_g}(x_{gi} - \Bar{x})(x_{gi} - \Bar{x})^T = n_g (\Bar{x}_g -\Bar{x})(\Bar{x}_g -\Bar{x})^T + \sum_{i=1}^{n_g}(x_{gi} -\Bar{x}_g)(x_{gi} -\Bar{x}_g)^T
\end{eqnarray*}

\subsection{Step 5}
If we further sum over $g=1,2,...,G$, we have\\
\begin{eqnarray*}
\underbrace{\sum_{g=1}^G\sum_{i=1}^{n_g}(x_{gi} - \Bar{x})(x_{gi} - \Bar{x})^T}_\text{$SSCP_{tot}$: total SSCP} = \underbrace{\sum_{g=1}^G n_g (\Bar{x}_g -\Bar{x})(\Bar{x}_g -\Bar{x})^T}_\text{$SSCP_{tr}$: treatment SSCP} + \underbrace{\sum_{g=1}^G \sum_{i=1}^{n_g}(x_{gi} -\Bar{x}_g)(x_{gi} -\Bar{x}_g)^T}_\text{$SSCP_{res}$: residual SSCP}
\end{eqnarray*}
where "SSCP" is short for "sum of squares and cross products". Now we can eventually construct the MANOVA table: \\
\ \\ 
\begin{center}
  \begin{tabular*}{\textwidth}{@{\extracolsep{\fill}} ccccc @{}} \hline \hline
    & Source of variation & SSCP & df & \\ \hline 
    & Treatments & $B=\sum_{g=1}^G n_g (\Bar{x}_g -\Bar{x})(\Bar{x}_g -\Bar{x})^T$ & $G-1$  & \\
    & Residuals & $W=\sum_{g=1}^G \sum_{i=1}^{n_g}(x_{gi} -\Bar{x}_g)(x_{gi} -\Bar{x}_g)^T$ & $\sum_{g=1}^G n_g - G$  & \\ \hline
    & Total & $B+W=\sum_{g=1}^G\sum_{i=1}^{n_g}(x_{gi} - \Bar{x})(x_{gi} - \Bar{x})^T$ & $\sum_{g=1}^G n_g - 1$  & \\ \hline \hline
  \end{tabular*}
\end{center}

\newpage
\subsection{Some Remarks}
\begin{itemize}
    \item $B+W=\sum_{g=1}^G\sum_{i=1}^{n_g}(x_{gi} - \Bar{x})(x_{gi} - \Bar{x})^T$ is the total SSCP, that is, if we merge all the groups together and see it as \textbf{one sample with size} $n=\sum_{g=1}^G n_g$, $B+W$ is just the SSCP of this sample.\\
    
    \item "W" is short for "within". $W=\sum_{g=1}^G \sum_{i=1}^{n_g}(x_{gi} -\Bar{x}_g)(x_{gi} -\Bar{x}_g)^T$ is the residual SSCP, it can be equivalently expressed as:\\
    \begin{eqnarray*}
    W = \sum_{g=1}^G SSCP_g = \sum_{g=1}^G (n_g -1)S_g
    \end{eqnarray*} \\
    
    \item "B" is short for "between". $B=\sum_{g=1}^G n_g (\Bar{x}_g -\Bar{x})(\Bar{x}_g -\Bar{x})^T$ is the treatment SSCP, it equals the total SSCP of an imaginary sample with size $n=\sum_{g=1}^G$:\\
    \begin{eqnarray*}
    \underbrace{\Bar{x}_1,\Bar{x}_1,...,\Bar{x}_1}_{n_1},\underbrace{\Bar{x}_2,\Bar{x}_2,...,\Bar{x}_2}_{n_2},...,\underbrace{\Bar{x}_G,\Bar{x}_G,...,\Bar{x}_G}_{n_G},
    \end{eqnarray*}
\end{itemize}

\subsection{Wilk's Lambda Statistic}
The test statistic in MANOVA is the Wilk's lambda statistic: \\
\begin{eqnarray*}
\Lambda^* = \frac{det(W)}{det(W+B)}
\end{eqnarray*}

\subsection{Bartlett's Approximation}
If $n=\sum_{g=1}^G n_g$ is large, we have the Bartlett's approximation: \\
\begin{eqnarray*}
-\left( n-1-\frac{p+G}{2} \right)ln \Lambda^* \sim \chi^2 (p(G-1))
\end{eqnarray*}
under $H_0: \mu_1 = \mu_2 = ... = \mu_G$.

\end{document}